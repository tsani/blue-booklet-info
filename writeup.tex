\documentclass[11pt]{article}

\usepackage[margin=2.0cm]{geometry}

\author{Jacob Errington \& Eric Mayhew}
\title{BLUE booklet information}

\newcommand{\gametitle}{\textsc{Sixteen-Fifty}}

\begin{document}

\maketitle

\section{Bios}
\begin{description}
\item[Jake.] ~ \\
  Jake first got excited about programming as a teenager when a friend told him
  they figured out how to make games. Jake's interests in programming have
  fluctuated over the years, but games and game development remain nonetheless
  an object of passion for him. In particular, he wonders how games can be used
  in education.

\item[Eric.] ~
\end{description}

The BLUE fellowship gave us both the opportunity to explore the use of games to
teach and to excite high-schoolers about the early history a New France. This
was a fantastic learning experience for the both of us: neither of us had built
a game before -- at least not one of this complexity!

\section{Project}

\subsection{Motivation}

The working title of our game, \gametitle, reflects in part the time period
in which it is set: the early colonial history of New France. But why choose to
focus on this time period specifically?
\begin{description}
  \item[Standardized testing.]
    In Quebec, all high school students must pass a standardized test on Quebec
    history, which includes a section on New France. As with the other tests,
    the history test has a startling failure rate, which we believe can be
    ameliorated by using novel teaching methods. Simply put, kids like games;
    it's hard even to get the off their phones in class! So why not meet them at
    their level, with a way to use their phones for educational purposes?
  \item[Historical revisionism.]
    The early history of New France is typically presented in a way that
    underrepresents the presence and importance of Indigenous Peoples.
    For example, it surprises many when we tell them that colonial settlements
    typically had Indigenous people walking around them. Another underemphasized
    occurrence is the annual trade fair in Ville-Marie (Montreal) in which
    colonists and Indigenous people took part both in trade and in celebration.
    Therefore, one aim of our game is to convey some of the additional
    complexities of the collaborations and conflicts between Indigenous Peoples
    and colonists.
\end{description}
%
To address these two primary concerns, we chose to design \gametitle{} as a
\emph{role-playing game} (RPG), in which the player controls a character through
whom they can experience the early history of New France first-hand.  The player
completes a series of \emph{quests}, some of which represent major historical
events and some of which represent additional events we simply thought were
interesting.

\subsection{Implementation}

Making a game requires work on many fronts, which we distilled into three
categories: the art, the story, and the engine.

Jake took charge of programming the game engine.
The game engine defines all the semantic concepts in the game,
such as the kinds of worlds the player can explore, the entities that exist in
these worlds, and the types of interactions the player character can have with
those entities. Furthermore, it's essential to build \emph{editors} for these
things: as game designers, we need high-level tools to manipulate worlds,
characters, and interactions without needing to dive directly into the game's
code. We're very proud of the \emph{event editor} in figure~1. This editor
ultimately evolved an \emph{embedded programming language} complete with
conditions and the ability to chain \emph{event scripts} together.

Eric took charge of the art and the story.
We chose a style called \emph{pixel art} on the one hand because we personally
like it, and on the other hand because we thought that the constraints it
imposes would help us to be productive. In figure~2, you can see a part of the
Quebec City map that Eric created. This map is historically accurate: in
figure~3 you can see a historical diagram of Quebec City's lower town, which in
the basis for figure~2.

\subsection{Future work}

\gametitle{} is not finished. In retrospect we were naive to think we could
arrive at a finished game in no more than eight weeks! What's left is to
translate the history into a series of main quests and side quests and to create
tons more art. As for the game engine, even expressing simple quests is quite
challenging in the embedded programming language, so some work will be needed to
soften the edges a bit on it. We look forward to tackling the next steps and to
someday have students playing and learning from our game!

\end{document}
